%%%%%%%%%%%%%%%%%%%%%%%%%%%%%%%%%%%%%%%%%%%%%%%%%%%%%%%%%%%%%%%%%%%%%%
% writeLaTeX Example: A quick guide to LaTeX
%
% Source: Dave Richeson (divisbyzero.com), Dickinson College
% 
% A one-size-fits-all LaTeX cheat sheet. Kept to two pages, so it 
% can be printed (double-sided) on one piece of paper
% 
% Feel free to distribute this example, but please keep the referral
% to divisbyzero.com
% 
%%%%%%%%%%%%%%%%%%%%%%%%%%%%%%%%%%%%%%%%%%%%%%%%%%%%%%%%%%%%%%%%%%%%%%
% How to use writeLaTeX: 
%
% You edit the source code here on the left, and the preview on the
% right shows you the result within a few seconds.
%
% Bookmark this page and share the URL with your co-authors. They can
% edit at the same time!
%
% You can upload figures, bibliographies, custom classes and
% styles using the files menu.
%
% If you're new to LaTeX, the wikibook is a great place to start:
% http://en.wikibooks.org/wiki/LaTeX
%
%%%%%%%%%%%%%%%%%%%%%%%%%%%%%%%%%%%%%%%%%%%%%%%%%%%%%%%%%%%%%%%%%%%%%%

\documentclass[10pt,landscape]{article}
\usepackage{amssymb,amsmath,amsthm,amsfonts}
\usepackage{multicol,multirow}
\usepackage{calc}
\usepackage{ifthen}
\usepackage[landscape]{geometry}
\usepackage[colorlinks=true,citecolor=blue,linkcolor=blue]{hyperref}


\ifthenelse{\lengthtest { \paperwidth = 11in}}
    { \geometry{top=.5in,left=.5in,right=.5in,bottom=.5in} }
	{\ifthenelse{ \lengthtest{ \paperwidth = 297mm}}
		{\geometry{top=1cm,left=1cm,right=1cm,bottom=1cm} }
		{\geometry{top=1cm,left=1cm,right=1cm,bottom=1cm} }
	}
\pagestyle{empty}
\makeatletter
\renewcommand{\section}{\@startsection{section}{1}{0mm}%
                                {-1ex plus -.5ex minus -.2ex}%
                                {0.5ex plus .2ex}%x
                                {\normalfont\large\bfseries}}
\renewcommand{\subsection}{\@startsection{subsection}{2}{0mm}%
                                {-1explus -.5ex minus -.2ex}%
                                {0.5ex plus .2ex}%
                                {\normalfont\normalsize\bfseries}}
\renewcommand{\subsubsection}{\@startsection{subsubsection}{3}{0mm}%
                                {-1ex plus -.5ex minus -.2ex}%
                                {1ex plus .2ex}%
                                {\normalfont\small\bfseries}}
\makeatother
\setcounter{secnumdepth}{0}
\setlength{\parindent}{0pt}
\setlength{\parskip}{0pt plus 0.5ex}
% -----------------------------------------------------------------------

\title{Quick Guide to LaTeX}

\begin{document}

\raggedright
\footnotesize

\begin{center}
     \Large{\textbf{A quick guide to \LaTeX}} \\
\end{center}
\begin{multicols}{3}
\setlength{\premulticols}{1pt}
\setlength{\postmulticols}{1pt}
\setlength{\multicolsep}{1pt}
\setlength{\columnsep}{2pt}


\section{code}
These are the personallized code for use of this system.
\verb!\\!
\subsection{Changes to Default System}

\begin{center}
\begin{tabular}{ c c }
C-x C-c  & unbind kill key 
\end{tabular}
\end{center}

\subsection{References Managment}
\begin{center}
\begin{tabular}{ c c }
C-c f & helm-bibtex-with-local-bibliography \\
C-u C-c f & helm-bibtex-with-local-bibliography - To refresh bibtex key list
\end{tabular}
\end{center}

\subsection{Navigation}
\begin{center}
\begin{tabular}{ c c }
C-c s & helm-buffers-list \\
C-c o & browse-file-directory - Command to open the directory of current file.
\end{tabular}
\end{center}

\subsection{File Interaction/Managment}
\begin{center}
\begin{tabular}{ c c }
C-c m & compile. Useful for Makefile commands \\
C-x g & magit-status
\end{tabular}
\end{center}

\subsection{Org-mode}
\begin{center}
\begin{tabular}{ c c }
C-c a & org-agenda - \\
C-c c & org-capture - \\
C-c j & gs-helm-org-link-to-contact - \\
C-c e & my-org-export-url - Pull external links into clipboard. \\
TBS & my-org-copy-smart-url -
\end{tabular}
\end{center}

\subsection{Projectile}
\begin{center}
\begin{tabular}{ c c }
s-p & projectile-command-map - \\
C-c p & projectile-command-map -
\end{tabular}
\end{center}

\end{multicols}

\end{document}
